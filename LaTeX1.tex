\documentclass[]{article}

%These tell TeX which packages to use.
\usepackage{array,epsfig}
\usepackage{amsmath}
\usepackage{amsfonts}
\usepackage{amssymb}
\usepackage{amsxtra}
\usepackage{amsthm}
\usepackage{mathrsfs}
\usepackage{color}
\usepackage[latin1]{inputenc}
\usepackage[cyr]{aeguill}
\usepackage[francais]{babel}
\usepackage{lmodern}
\let\oldquote\quote
\let\endoldquote\endquote
\renewenvironment{quote}[2][]
  {\if\relax\detokenize{#1}\relax
     \def\quoteauthor{#2}%
   \else
     \def\quoteauthor{#2~---~#1}%
   \fi
   \oldquote}
  {\par\nobreak\smallskip\hfill(\quoteauthor)%
   \endoldquote\addvspace{\bigskipamount}}

%Here I define some theorem styles and shortcut commands for symbols I use often
\theoremstyle{definition}
\newtheorem{defn}{D�finition}
\newtheorem{princ}{Principe}
\newtheorem{thm}{Th�or�me}
\newtheorem{cor}{Corollary}
\newtheorem*{rmk}{Remarque}
\newtheorem{lem}{Lemma}
\newtheorem*{joke}{Joke}
\newtheorem{ex}{Exemple}
\newtheorem*{soln}{Solution}
\newtheorem{prop}{Proposition}
\newtheorem{exo}{Exercice}
\newtheorem*{ph}{Parenth�se Historique}

\newcommand{\lra}{\longrightarrow}
\newcommand{\ra}{\rightarrow}
\newcommand{\surj}{\twoheadrightarrow}
\newcommand{\graph}{\mathrm{graph}}
\newcommand{\bb}[1]{\mathbb{#1}}
\newcommand{\Z}{\bb{Z}}
\newcommand{\Q}{\bb{Q}}
\newcommand{\R}{\bb{R}}
\newcommand{\C}{\bb{C}}
\newcommand{\N}{\bb{N}}
\newcommand{\M}{\mathbf{M}}
\newcommand{\m}{\mathbf{m}}
\newcommand{\MM}{\mathscr{M}}
\newcommand{\HH}{\mathscr{H}}
\newcommand{\Om}{\Omega}
\newcommand{\Ho}{\in\HH(\Om)}
\newcommand{\bd}{\partial}
\newcommand{\del}{\partial}
\newcommand{\bardel}{\overline\partial}
\newcommand{\textdf}[1]{\textbf{\textsf{#1}}\index{#1}}
\newcommand{\img}{\mathrm{img}}
\newcommand{\ip}[2]{\left\langle{#1},{#2}\right\rangle}
\newcommand{\inter}[1]{\mathrm{int}{#1}}
\newcommand{\exter}[1]{\mathrm{ext}{#1}}
\newcommand{\cl}[1]{\mathrm{cl}{#1}}
\newcommand{\ds}{\displaystyle}
\newcommand{\vol}{\mathrm{vol}}
\newcommand{\cnt}{\mathrm{ct}}
\newcommand{\osc}{\mathrm{osc}}
\newcommand{\LL}{\mathbf{L}}
\newcommand{\UU}{\mathbf{U}}
\newcommand{\support}{\mathrm{support}}
\newcommand{\AND}{\;\wedge\;}
\newcommand{\OR}{\;\vee\;}
\newcommand{\Oset}{\varnothing}
\newcommand{\st}{\ni}
\newcommand{\wh}{\widehat}

%Pagination stuff.
\setlength{\topmargin}{-.3 in}
\setlength{\oddsidemargin}{0in}
\setlength{\evensidemargin}{0in}
\setlength{\textheight}{9.in}
\setlength{\textwidth}{6.5in}
\pagestyle{empty}



\begin{document}

\section{D�riv�es des fonctions usuelles}

\begin{equation*}
(x^n)' = n \cdot x^{n-1}
\end{equation*}

\begin{equation*}
\cos'(x) = -\sin(x)
\end{equation*}

\begin{equation*}
\sin'(x) = \cos(x)
\end{equation*}

\begin{equation*}
(e^x) ' = e^x 
\end{equation*}

\begin{equation*}
\ln'(x) = \frac{1}{x}
\end{equation*}

\section{Op�ration autour de la d�rivation}

\begin{thm}
Soit $f$, $g$ : $ \mathbb{R} \rightarrow \mathbb{R}$ deux fonctions et $a\in \mathbb{R}$. On as les formules suivantes :
\begin{enumerate}
\item[(i)] \textbf{Formule de Leibnitz}
\begin{equation*}
(f\cdot g) ' (x) = f'(x) \cdot g(x) + g'(x) \cdot f(x)
\end{equation*}
\item[(ii)] \textbf{Lin�arit�}
\begin{equation*}
(f + g)' (x) = f'(x) + g'(x) \hspace{0.7em}\text{\&} \hspace{0.7em} (a\cdot f)'(x) =  a\cdot f'(x)
\end{equation*}
\item[(iii)] \textbf{Formule de composition}
\begin{equation*}
(f\circ g)'(x) = g'(x) \cdot f'(g(x))
\end{equation*}

\item[(iii)] \textbf{D�riv�e de l'inverse}, on suppose $f$ est non nul.
\begin{equation*}
(\frac{1}{f})'(x) = -\frac{f'(x)}{f(x)^2}
\end{equation*}
\end{enumerate}
\end{thm}

\section{Exercices}

Calculez les d�riv�es des fonctions suivantes: 
\begin{equation*}
f(x) = 3\cdot x^4 + 1
\end{equation*}

\begin{equation*}
f(x) = x \cdot e^{\cos(x)}
\end{equation*}

\begin{equation*}
f(x) = \ln(1+x^2)
\end{equation*}

\begin{equation*}
f(x) = \frac{x^2 + 3}{1+x^4}
\end{equation*}

\section{Les elements diff�rentielle en physique}

Montrons les formulles avec des outils intuitif. Commen�ons par le (ii). Si $x$ varie de $\Delta x$ alors on note $\Delta f$ la variation de $f$ entre $x$ et $x+\Delta x$. De m�me pour $g$ et $f+g$. On as alors : 
\begin{equation*}
\begin{align*}
\Delta(f + g) & = f(x+\Delta x) + g(x+\Delta x) -  (f(x) + g(x)) \\
& = \big(f(x+\Delta x) - f(x)\big) + \big(g(x+\Delta x) - g(x)\big)\\
& = \Delta f + \Delta g 
\end{align*}
\end{equation*}
en divisant par $dx$ des deux cot� :
\begin{equation*}
\frac{\Delta(f + g)}{\Delta x} = \frac{\Delta f}{\Delta x} + \frac{\Delta g}{\Delta x} 
\end{equation*}

en prennant la limite quand $\Delta x$ tend vers $0$ on obtient : 
\begin{equation*}(f + g)' (x) = f'(x) + g'(x)\end{equation*}

Pour le (i)
\begin{equation*}
\begin{align*}
\Delta (f g) & = f(x+\Delta x)g(x+ \Delta x) - f(x)g(x) \\
& = (f(x) + \Delta f)(g(x) + \Delta g) - f(x)g(x) \\
& = f(x)g(x)+g(x)\Delta f+f(x)\Delta g + \Delta g \Delta f - f(x)g(x) \\
& = g(x)\Delta f + f(x) \Delta g + \Delta g \Delta f \\
\frac{\Delta(fg)}{\Delta x} &= g(x)\frac{\Delta f}{\Delta x} + f(x)\frac{\Delta g}{\Delta x} + \frac{\Delta f \Delta g}{\Delta x}
\end{align*}
\end{equation*}
Ici il faut comprendre que dans la d�riv� seule les terme <<d'ordre 1>> compte, c'est � dire ceux qui n'ont qu'un $d$. Autrement on peut v�rifier que Le terme $\frac{\Delta f \Delta g}{\Delta x}$ tend vers 0. On trouve donc finalement : 
\begin{equation*}
(fg)'(x) = g(x)f'(x) + f(x)g'(x)
\end{equation*}


\end{document}